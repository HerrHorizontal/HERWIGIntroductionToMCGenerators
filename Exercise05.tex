\documentclass[a4paper,12pt]{article}
\usepackage{../common/style_exercise}
\input{../common/definitions}
\usepackage[nameinlink]{cleveref}

\begin{document}
    \names
    \header{5}{December 01, 2022} % put here # of exercise (argument 1) and date to hand in exercise (argument 2)

    \section*{Event Generation with Herwig7}

    \subsection*{\Herwig}

    \hyperref{https://herwig.hepforge.org/}{}{}{Herwig} is a multi-purpose particle physics event generator.
    It is composed of many modules, which have been developed over many years by various authors.
    In the current version it is written in C++.
    Many efforts have been done to improve and develop the description of particle collisions, which involve heavy theoretical and phenomenological research of perturbative and non-perturbative Quantum Field Theory (QFT) and empirical models.
    One of the \hyperref{https://www.itp.kit.edu/memberpages/gieseke}{}{}{reasearch groups of \Herwig lead by Stefan Gieseke} is based here at the KIT.
    \\\\
    Multi-purpose (or general-purpose) event generators like \Herwig\footnote{Other multi-purpose generators are \hyperref{http://home.thep.lu.se/Pythia/}{}{}{\Pythia} or \hyperref{https://sherpa-team.gitlab.io/}{}{}{\Sherpa}, which provide similar functionalities, but involve in many cases slightly different models.} can be used to generate events of a variety of colliding particles using various types of Monte-Carlo (MC) algorithms in several stages.
    These are based on perturbative and non-perturbative QFTs or motivated by empirical models, which aim to describe observed phenomena and characteristic properties of high energy collisions.
    This exercise is built with the notion to give you a feeling for the necessary generation steps, in our case implemented in \Herwig, to achieve a good prediction for measureable data, without the necessity to understand all technical and theoretical details.


    \subsection*{\Rivet}

    \hyperref{https://rivet.hepforge.org/}{}{}{\Rivet} is a C++ based toolkit, which enables the implementation of robust and efficient analyses of generator outputs.
    Such analysis, provided by the analyzers of the corresponding experiments, are mainly used for comparison of generated predictions with unfolded measurements.
    Therefore they can be used for generator development, validation and tuning of free model parameters.
    Moreover, it provides an easy and convenient infrastructure for analyzing generated particle-collision events, which we will use in this exercise.


    \subsection*{Getting Started}

    Open the \href{https://jupytermachine.etp.kit.edu}{jupytermachine} for the exercises and start a \texttt{Herwig} server. Once loaded, update your local copy of the \href{https://gitlab.etp.kit.edu/Lehre/tp1_forstudents}{tp1\_forstudents} repository and work through the Jupyter Notebook inside the \texttt{Exercise05} folder.

\vfill

\hrule
  \begin{center}
    \includegraphics[width=\textwidth]{Vineta_Poster_Banner.pdf}
  \end{center}
  \noindent
  \hrule


\end{document}
